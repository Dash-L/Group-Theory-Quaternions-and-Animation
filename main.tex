\documentclass[12pt, letterpaper]{article}
% Just for better margins
\usepackage[margin=1.0in, includehead, bmargin=1.166667in, headsep=0pt]{geometry}
\usepackage{amsmath}
\usepackage{amsfonts}
\usepackage{graphicx} % Required for inserting images

\usepackage{biblatex}

\addbibresource{bibliography.bib}

\setlength\parskip{0.75em}
\setlength\parindent{0em}

\title{Quaternions and Animation}
\author{Dashiell Elliott, Luke Harper, Robin Husman, Will Sakalauskas}
\date{December 2024}

\renewcommand{\i}{\mathbf{i}}
\renewcommand{\j}{\mathbf{j}}
\renewcommand{\k}{\mathbf{k}}
\newcommand{\R}{\mathbb{R}}
\newcommand{\bbR}{\mathbb{R}}
\newcommand{\bbZ}{\mathbb{Z}}
\DeclareMathOperator\GL{GL}
\DeclareMathOperator\SL{SL}
\DeclareMathOperator\SO{SO}
\renewcommand{\H}{\mathbb{H}}
\newcommand{\bbH}{\mathbb{H}}
\renewcommand{\b}[1]{\mathbf{#1}}

\begin{document}

\maketitle

\section{Introduction}
How can we measure and perform calculations with rotations of 3-dimensional space?
It's not hard to imagine that we would want to use a computer to rotate points
in a CAD program or for some animation software or a video game.
In 2-dimensions, we have essentially 3 options: \begin{enumerate}
    \item Rotations can be stored as an angle in $[0,2\pi)$,
        which can be treated as representatives for the factor group $\bbR/2\pi\bbZ$.
    \item Since rotations are linear transformations,
        they can be stored as $2\times2$ matrices of real numbers,
        more specifically in the group of orthogonal determinant $1$ matrices,
        $\SO(2) \le \SL_2(\bbR) \le \GL_2(\bbR)$.
    \item Multiplication by complex numbers of magnitude $1$ rotates the complex plane,
        so we can store the rotations as complex numbers.
\end{enumerate}
These methods have different advantages in different contexts.
Angles are easy to read but require some trigonometry to actually apply,
while matrices and complex numbers are both harder to interpret but
generally more friendly to work with computationally and theoretically.
Similarly, we have options when it comes to representing 3-dimensional rotations. \begin{enumerate}
    \item We can extend the angle technique to 3 dimensions by storing rotations
        as yaw, pitch, and roll, making angle triplets called ``Euler angles''.
    \item We can also deal with matrices by just adding a dimension and using the
        $3\times3$ matrices of $\SO(3)$.
    \item A complex number with more dimensions is\ldots\ what exactly?
\end{enumerate}
This is the idea behind quaternions, and it's what we'll be explaining in the rest of this report.
Euler angles are very inconvenient for computation and while matrices are a significant upgrade,
they still require storing $9$ numbers and turn out not to be nearly as nice as quaternions
in some ways.

\section{Motivation}
Since the complex numbers allow us a way to represent rotations in 2 dimensions, it is sensible to look for a 3 dimensional variant that will represent rotations in 3 dimensions. Unfortunately, no such algebra exists \cite{frobenius1877}.

\section{Group Theory of the Quaternions}
Generally, quaternions are a ring -- which is a set equipped with both an addition and a multiplication -- denoted $\H$. Here we will focus mainly on the  quaternions as a group under the operation of quaternion multiplication (however we will still call the group $\H$).

One important subgroup of $\H$ is $Q_8 = \{1, i, j, k, -1, -i, -j, -k\}$, the set of the basis quaternions in each axis. It has six subgroups:
\begin{enumerate}
    \item $\{1\}$
    \item $\{1,-1\} = \langle -1 \rangle$
    \item $\{1, -1, i, -i\} = \langle i \rangle$
    \item $\{1, -1, j, -j\} = \langle j \rangle$
    \item $\{1, -1, k, -k\} = \langle k \rangle$
    \item $Q_8$
\end{enumerate}

The center is $Z(Q_8) = \{1, -1\} = \langle 1 \rangle$ and $Q_8 / Z(Q_8) \approx \mathbb{Z}_2 \oplus \mathbb{Z}_2$

The group $Q_8$ is Hamiltonian, that is, it is not abelian, but every subgroup is normal.

Another subgroup is the set of unit quaternions, which we will call $V$. That is, $V=\{a+b\i+c\j+d\k: a,b,c,d \in \R \text{ with }a^2+b^2+c^2+d^2=1\}$. We claim that $V$ is a subgroup of $\H^*$. Clearly, the identity element $1$ is in $V$ since $1^2+0^2+0^2+0^2=1$. So, $V$ is nonempty. We omit the (vastly many lines of) algebra to verify that the product of two unit quaternions is a unit quaternion, and that rational powers of unit quaternions are unit quaternions.

%[Something about the inverses? Idk]

% Something about representation as cos theta + u sin theta
% Note that conjugate is cos theta - u sin theta
% Note that square root is cos theta/2 + u sin theta/2

\section{Interpretation of Unit Quaternions as Rotation Around an Axis}

By Euler, summarized in \cite{kuipers}, ``Any two independent orthonormal coordinate frames may be related by a
single rotation about some axis.''

Let us see if we can discover how this idea relates to unit quaternions!

By \cite{stackex-polar}, for any unit quaternion $q$, we can choose a real number $\theta$ and a pure unit quaternion $\b{u}$ such that $q=\cos \theta + \b{u} \sin \theta$.

Observe that $\b{u}^2=(a\i+b\j+c\k)(a\i+b\j+c\k)=(-a^2-b^2-c^2+ab\i\j+ab\j\i + ac\i\k+ac\k\i+bc\j\k+bc\k\j)=-1$.

So, for any real $\alpha$,
\begin{align*}
    q (\alpha \b{u})q^* &= (\cos \theta + \b{u} \sin \theta)(\alpha \b{u})(\cos \theta - \b{u} \sin \theta) \\
    &= (\cos \theta + \b{u} \sin \theta)(\alpha \b{u} \cos \theta - \alpha \b{u} \b{u} \sin \theta) \\
    &= \alpha (\cos \theta + \b{u} \sin \theta)( \b{u} \cos \theta + \sin \theta) \\
    &= \alpha (\b{u} \cos^2 \theta + \cos\theta \sin \theta + \b{u} \b{u} \sin \theta \cos \theta + \b{u} \sin^2 \theta) \\
    &= \alpha(\b{u}(\cos^2 \theta + \sin^2 \theta) + 0) \\
    &= \alpha \b{u}
\end{align*}

That is, the rotation represented by $q$ takes any point in the line in $\R^3$ spanned by $\b{u}$ to itself. Since no assumptions about $q$ other than magnitude were made, this must mean the pure unit quaternion $\b{u}$ is in the direction of the axis about which the rotation is made.

It is perhaps a reasonable assumption by our notation that $\theta$ may represent the angle by which $q$ rotates vectors; This assumption is correct up to a factor of one half \cite{hanson}, though the proof is omitted here for brevity.

\section{Advantages of Unit Quaternions as Rotations (Over Euler Angles)}

\subsection{Interpolation}

Intuitively, there is a very easy way to apply half of a rotation represented as a quaternion. Let $f$ be a rotation. For some $q\in V$, we can write $f$ as $f(p)=qpq^*$ for all pure quaternions $p$. That is, $f$ is the rotation represented by $q$. Then, consider the representation represented by $q^{1/2}$. We will call this rotation $g$, defined by $g(p)=q^{1/2}p(q^{1/2})^*$. This is the functional square root of $f$; $f(p)=qpq^*=q^{1/2}q^{1/2}p(q^{1/2})^*(q^{1/2})^*=g(g(p))$. So, in some sense, $g$ is half of the rotation of $f$.

Let us choose pure unit quaternion $u$ and real angle $\theta$ such that $q=\cos \theta + \b{u} \sin \theta$. Then $q^{1/2}=\cos \frac{\theta}{2} + \b{u} \sin \frac{\theta}{2}$.

This technique applies more generally for any interpolation time coordinate $t \in [0,1]$ --- applying $q^t=\cos{t\theta} + \b{u} \sin{t\theta}$ --- in what \cite{hanson} calls SLERP.

\section{}

\medskip

\printbibliography

\end{document}
